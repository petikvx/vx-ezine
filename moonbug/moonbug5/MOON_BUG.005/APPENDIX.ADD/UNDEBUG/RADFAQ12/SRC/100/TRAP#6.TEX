From Andrew Kohlsmith  -  Anti-Debugging Tricks.

Here's a couple I found..  They aren't MY ideas, but they're good for
dissasseblers nontheless:  Hiding instructions in instructions.

erp:    mov     ax,0FE05h
        jmp     $-2h
        add     ah,03Bh
        ...                     ; rest of code

Let us simulate what would happen if we were to trace through this code,
showing a hex dump at each step to clarify things.

 B8 05 FE EB FC 80 C4 3B   mov     ax,0FE05h       ; ax=FE05h
 ^^ ^^ ^^
 B8 05 FE EB FC 80 C4 3B   jmp     $-2             ; jmp into '05 FE'
          ^^ ^^
 B8 05 FE EB FC 80 C4 3B   add     ax,0EBFEh       ; 05 is 'add ax'
    ^^ ^^ ^^
 B8 05 FE EB FC 80 C4 3B   cld                     ; a dummy instruction
             ^^
 B8 05 FE EB FC 80 C4 3B   add     ah,3Bh          ; ax=2503h
                ^^ ^^ ^^

     The add ah,03Bh is there simply to put the value 2503h into ax.  By adding
five bytes (as opposed to simply using 'mov ax,2503h') this code will confuse
disassemblers pretty well.  Even if the instructions are disassembled properly,
the value of ax will not be known, so every int call after this point will not
be commented properly, as long as you never move a value into ax.  You can
conceal the value from the disassembler by using 'add ax' or 'sub ax' whenever
possible.

     If you examine this closely, you can see that any value can be put into
ax.  Two of the values can be changed to whatever you want, namely the FE in
the first line, and the 3B in the last line.  It is helpful to debug through
this chunk of code to determine what values should be placed here in order to
make ax what you would like it to be.

     Back to the subject of killing debuggers, it is very sneaky to hide
something like a hlt instruction inside another instruction, such as a jmp.
For example, take a look at this:

glurb:  mov     cx,09EBh
        mov     ax,0FE05h ;-\
        jmp     $-2       ;  >--this should look familiar to you
        add     ah,03Bh   ;-/
        jmp     $-10
        ...             ; rest of code

The three lines in the middle are a repeat of the previous example.  The
important part of this code is the first line and the 'jmp $-10'.  What happens
is, the jmp goes back into the 'mov cx' instruction.  The '09EB' translates
into 'jmp $+9'.  This lands in the '$-10' part of the first jmp.  The $-10 just
happens to be stored as 0F4h, the hlt instruction.  By making the hlt part of
another instruction, it is not visible when it is being traced through by
td386.  It is also not possible to remove it without altering the code.

Of course, for HLT to be really effective, you should hide a out 21h, ff and
out a1h, ff...  :-)

TZ

